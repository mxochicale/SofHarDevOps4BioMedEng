\documentclass{article}
\usepackage{datetime2}

\title{
%Reproducible biomedical engineering workshops (ReBiEng): 
%From data, code, cad to ml pipelines 
% Sat  9 Jul 12:11:39 BST 2022
%Hands-on Workshops for Biomedical Engineers on Software/Hardware Development Toolkits %Sat  9 Jul 12:25:39 BST 2022
SofHarDevOps4BiomedicalEng: Hands-on Workshop on Software and Hardware Developer Operations for Biomedical Engineers %Mon 11 Jul 17:10:46 BST 2022
}

\author{Miguel Xochicale}
%\textit{et al.}}
\date{\DTMNow}
%(400words)

\begin{document}
\maketitle

Since the publication of the spectrum of reproducubility by Roger Peng in December 2011, there has been a good response from the scientific community to adapt the full scientific replication, including code, data and software \cite{peng2011}. 
%published a roadmap for open and reproducible science going from publication only to 
However, there is a current need to adapt the best practices of fully scientific replication workflows to the combination of software and hardware and hardware itself.
Diederich \textit{et al.} \cite{Diederich2022}, for instance, highlighted the challenges of proper file-sharing and policies for hardware, leading authors to propose the use of guidelines like OSHWA and workshops for Open Hardware Makers as way to create more trustworthy science. 
Similarly, Stirling et al. \cite{stirling2022} proposed hardOps, hardware operations, as way to address challenges in reproducibility for hardware, consisting on six stages: plan, design and document, prepare and verify, distributed production, physical testing, and feedback. 
Following up previous our work on "open-corTeX: A framework for Continuously-integrated Open-source Reproducible TeX"
\cite{xochicale2020}, this work presents a workshop for the best practices for Software and Hardware operations, aiming to train the next generation of Biomedical Engineers.
The workshops contains topics on (1) introduction to git, github, (2) project management, (3) continues integration, (4) standards,(5) exercises and (6) examples of projects following good practices.
The resources for the workshop are available at https://github.com/mxochicale/SofHarDevOps4BiomedicalEng.

\newpage
%\bibliographystyle{apalike}
\bibliographystyle{plain}
\bibliography{references/references} %%GITHUB
%\bibliography{references} %%OVERLEAF

\end{document}
