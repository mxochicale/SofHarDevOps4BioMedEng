\documentclass{article}
\usepackage{datetime2}

%## OVERLEAF: "Anyone with this link can edit this project"
%https://www.overleaf.com/2673429979cmyzgqwvxhdj

\title{
%Reproducible biomedical engineering workshops (ReBiEng): 
%From data, code, cad to ml pipelines 
% Sat  9 Jul 12:11:39 BST 2022
%Hands-on Workshops for Biomedical Engineers on Software/Hardware Development Toolkits %Sat  9 Jul 12:25:39 BST 2022
%SofHarDevOps4BiomedicalEng: Hands-on Workshop on Software and Hardware Developer Operations for Biomedical Engineers %Mon 11 Jul 17:10:46 BST 2022
%SofHarDevOps4BioMedEng: Hands-on Workshop on Software and Hardware Developer Operations for Biomedical Engineers %Tue 19 Jul 06:37:30 BST 2022
Hands-on Workshop on Software and Hardware Developer Operations for Biomedical Engineers: SofHarDevOps4BioMedEng %Tue 19 Jul 21:12:44 BST 2022
}

\author{
Eric Kerfoot and Miguel Xochicale
}
%\textit{et al.}}
\date{\DTMNow}
%(400words)

\begin{document}
\maketitle
In last decade, there has been a good response from the scientific community to adopt the full scientific replication, including code, data, and software~\cite{peng2011}. 
%published a roadmap for open and reproducible science going from publication only to 
However, there is a current need to adopt the best practices of fully scientific replication workflows to both software and hardware.
Recently, Diederich \textit{et al.} 2022, for instance, highlighted the challenges of proper file-sharing and policies for hardware, which led Diederich \textit{et al.} to suggest using guidelines of Open Source Hardware Association (OSHWA) and workshops for Open Hardware Makers as way to create more trustworthy science~\cite{Diederich2022}.
Similarly, Stirling \textit{et al.} 2022 proposed hardOps, hardware operations, as a way to address challenges in reproducibility for hardware, consisting on six stages: plan, design and document, prepare and verify, distributed production, physical testing, and feedback~\cite{stirling2022}. 
Hence, following up our previous work on "open-corTeX: A framework for Continuously-integrated Open-source Reproducible TeX"~\cite{xochicale2020}, this work presents a workshop on the best practices for Software and Hardware operations, aiming to equipping the next generation of Biomedical Engineers with appropriate skills and tools to create reproducible and trustworthy science.
The workshop contains six lessons on (1) introduction to git, GitHub, (2) project management, (3) continuous integration, (4) standards, (5) exercises and (6) examples of projects.
%aligned to good practices for software and hardware.
The resources of the workshop are available at https://github.com/mxochicale/SofHarDevOps4BioMedEng.

\newpage
%\bibliographystyle{apalike}
\bibliographystyle{plain}
\bibliography{references/references} %%GITHUB
%\bibliography{references} %%OVERLEAF

\end{document}
